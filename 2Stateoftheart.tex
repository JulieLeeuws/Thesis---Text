\chapter{State of the art}

Sample text Sample text

\section{Animal/rodent kinematics}

Sample text Sample text

\section{OA animal models}

\subsection{What is OA ? \cite{OA1}}


Osteoarthritis (OA) is a multi-factorial degenerative disease present in the joint. It shows different symptoms which are: erosion of the cartilage in joints, bone formation at the edge of joints (osteophytosis), synovitis and even inflamation of the joint capsule. It is particularly present in joints which bear loads, like the knee, hips or back. It is separated into two types: primary and secondary. The secondary is the result of another disease which is anterior cruciate ligament (ACL) rupture. The primary on the other hand is a disease on its own and is more common, 82\% of OA patients suffer from primary OA. It develops gradually over time and can have many causes which are not always known.

In a normal joint, cartilage is regenerated by production of chondrocytes, their production is stimulated by the destruction of cartilage, which is caused by the enzyme matrix matalloproteinases (MMP's). But, if the destruction of the cartilage is faster than its production, erosion occurs and the cartilage begins to present cracks and can even break, leaving the lower layers of cartilage and even subchondral bone exposed and damaged. This is painful because bone contains nerve endings and blood vessels in opposition to cartilage which is aneural.

Current research to create new anti-arthritic medication consists of targeting the inhibition of the proteins responsible for the destruction of cartilage (MMP's). Other aspects are also considered, like controlling the mechanical factors and anti-inflammatory medication.

\subsection{Models \cite{OA1} \cite{OA2}}

There are several ways to modelize OA with the use of animals. The most common way to induce OA is by surgical destabilisation of the joint. The internal conditions of the joint change and OA is easily induced. This can be achieved with ACL transection or a medial meniscus tear for example. The issue with these is that only secondary OA is induced, so it is not yet clear if these models provide benefits in studying primary OA. Primary OA models have been used, but they tend to rely on genetically induced biological changes, but the change of inducing primary OA are slim and random. The disease also takes long to develop and this is not suited for large scale experiments. Recent studies have developed less invasive methods to destabilize the joint, like cyclic tibial compression, external joint loading of intra-articular fracture of subchondral bone. The advantage is that these avoid intra-articular surgery, and thus avoid distrubing the inside of the joint, resulting in primary OA.

\subsection{intra-articular surgery\cite{OA2}}

Here, OA is induced by surgically destabilising the medial meniscus, this involves going inside the joint and moving the medial meniscus medially. This can be performed by two different ways: by destabilisation of the medial meniscus (DMM) and anterior cruciate ligament transection (ACLT). \cite{DMM}

% DMM is executed by transecting the medial meniscotibial ligament (MMTL), which anchors the meniscus to the tibia plateau. By transecting it, the meniscus is destabilised

\subsection{non intra-articular surgery\cite{OA1}}

As said before, recent studies have shown that is it possible to induce primary OA by destabilizing the joint surgically, but without penetration of the joint. In the paper by Britzman et al., primary OA in induced in the knee of rodents by surgically inducing varus tibial malalignement, also know as "bowlegs". This causes increased loading on the medial side of the joint and leads to primary OA. It is based on a study by \cite{rabbitOA}, where the same technique was used on rabbits as this is a larger animal and the surgical procedure is made easier. There, a 30\degree valgus has been induced and has been shown to lead to mild degeneration of the cartilage which suggests an early stage of OA. But, these models are not representative of humans because of the species chosen. Rats have been showed to resemble more to humans than any other lab animal. \textcolor{red}{Need better explanation of why this is true}

The procedure has been done on Sprague-Dawley rats and consists of creating a 35\degree wedge osteotomy (removing a wedge of bone) just distal to the knee joint, the bone was secured with a metal plate bend to the angle of osteotomy. The sham group also underwent surgery but the osteotomy was of 0\degree , a metal plate was also used to secure the bone. The method to measure the kinematics was with the use of reflective markers, the rats were meant to walk through a 1,5m long walkway. OpenSim was used to make the musculoskeletal model and to determine the medial knee joint contact forces and adduction moments. A histologu was also performed on the knee joint. .

Histological grading showed that extensive loss of cartilage was present in the osteotomized animals, the difference were signicative between the osteotomized and sham animals. On the biomechanic side, there was also a significant difference between both groups for the medial knee joint contact force. The ground reactions forces remained the same, showing that the animals still load their limbs fully even with severe cartilage erosion.

With these analysis, following conclusions were drawn, the level of disease is significantly greater than in previous studies on rabbits. The level of erosion of the cartilage is higher, with deeper, wider and bigger lesions. This can be justified by the fact that rats are smaller animals and thus have thinner cartilage which develops bigger lesions earlier in the disease development. The fact that the animals still load their limbs fully during the experiment shows that it has good repeatability, because the change that OA is induced through biomechanical mechanisms is greater. The angle of the osteotomy can also be adjusted to create more or less medial loading. This can be controlled and keeps the model flexible for future studies.

\section{Existing techniques, methods to collect data}

Two main techniques are make rodents walk and to collect data for rodent kinematic analysis: open field tests and treadmills.\\

The open field technique consists of a large arena where the rodent can move freely while in treadmills the rodents are constricted to a small space and are forced to move at a certain speed in a straight line.

\subsection{Open field test}

The open field test consists of placing the rodent in a large arena, it can be a large squared arena or a narrower alley. The rodent them moves towards a goal or freely, depending on the type of information that is wanted.

Here, the analysis is done on the footsteps with consideration of the different parameters of the foot in order to reconstruct and evaluate the gait. While in the past, the methods to collect the paw prints were manual, nowadays digital and automated methods exist.\cite{agatha}\\

Manual methods consisted of using grease or Vaseline on white paper. The rodent would walk on the paper while their paws were coated with Vaseline. The paw prints would be made visible by dusting charcoal on the paper. It would adhere to the grease and reveal the footprints, the measurements where made with graph paper \cite{vaseline}. With the advancement of photography later on, the paper and Vaseline was replaced by photographic paper and photo developer \cite{photographicpaper}. Less toxic alternatives are later used, for example ink (or paint) and paper which remains the most common method. All of the mentioned techniques are widely used before the '80 and '90, but with the arrival of computers, other techniques will be developed in the area of paw print recognition. These consist of filming the mouse from the bottom and side. The setup is composed of a box with a clear floor where the rodent will walk on, lights to augment contrast, sometimes mirrors in order to see the lateral and ventral view with one camera image for example, and a camera. The images are then processed, sometimes manually and sometimes automatically \cite{agatha}.\\

Dijkstra et al. \cite{dijkstra} compared different methods of paw print acquisition in order to evaluate functional nerve recovery in adult rats. These methods were: photographic paper, finger paint and white paper, and video recordings. They found that photographic paper has disadvantages like the cost, the fact that photographic paper is slippery and causes smears and because the experiment had to be done in the dark, the gait could not be quantitatively evaluated. Finger paint and white paper are cheaper, less slippery and give the possibility to work in broad daylight. Finally, the advantage of using the video camera setup is that both ventral and lateral views can be seen at the same time and foot placing and the toe positions can easily be measured.\\

We now have an insight on this method. The measured information still needs to be processed and analysed quantitatively. First, a simple measuring tape would be used to measure distances between paw prints, especially in methods using paper. But now, automated software can detect the paw prints in the image, process them and give the wanted results as output.\\

One of the automated methods is AGATHA (Automated Gait Analysis Through Hues and Areas) \cite{agatha2}, it is an open source algorithm that identifies gait patterns of rats. This open-source analysis software is useful for researchers who don't want to buy commercial systems because of the cost and because the software is often unalterable. It is also useful for those who want to create their setup from scratch by buying an arena, which is relatively low cost, and don't want to analyse their data manually.
This is exactly why open source is ideal for this case, it is free, adaptable and it can be modified by the user base to correct methodological errors and to help optimising the software. As it is written in MATLAB\textsuperscript{\textregistered} code, it is quite accessible.\\

AGATHA works as follows: the software needs video images of the sagittal view and the ventral view of the walking rat. First, AGATHA subtracts a background image from the images with the rat, this is in order to isolate the silhouette of the rat in a sagittal view and to transform it into a HSV (Hues, Saturation, Value) image. The value of the Hue is used to convert the subtracted image into a binary image in black and white. Then, using this binary image, the interface between the paw and the floor is represented by a row of pixels, it is defined as the lowest row of pixels in the binary image. This creates a 2D line with white pixels (interface between paw and floor) and black pixels (background). This only works for rats which are walking and not running/trotting, because running/trotting involves an areal phase. So in this phase, the lowest row of pixels would not be an interface between the paw and the ground.

Second, all the nose and tail contacts with the ground are eliminated from the row of pixels, the particular pixels are determined by knowing the centre of the area of the sagittal silhouette. If the pixel is proportionally to far from this centre, the pixel is eliminated. Finally, each frame has a row of pixels and all these row are stacked foot contact over time can be visualised. Temporal events are localised, like to moment the paw touched the ground (foot-strike) and the moment the paw leaves the ground (toe-off), to extract different parameters of the gait. The parameters are: temporal symmetry of the hind limbs, duty factor of the hind and fore limbs, duty imbalance of the hind limbs, step frequency and limb phase between the fore and hind limbs.

Also, the location of the paw on the ventral view is determined with help of the sagittal view and the location of the foot-strike and the toe-off. The centroid of the paw is also determined with these two parameters. These spatial events are then used to determine the paw's spatial location during stance, the stride length, step widths, and step length symmetry.

This method of course has disadvantages, because the sagittal silhouette is determined by subtraction, the contrast has to be sufficient between the animal and the background for the software to works properly. The contrast is achieved with proper lighting of the arena. Also, the camera angle has to be set at the height of the floor, otherwise, reflection artefacts could appear in the image and falsify the analysis.

\subsection{Treadmill}

The treadmill method is more recent and provides different advantages than open field tests. Here, the kinematics are measured with the help of markers on the skin or anatomical land marks if we are working with x-ray cameras. The issue with markers on the skin is the skin movement error. Because of the loose skin, the marker moves freely relatively to the underlying bone and cause errors while the animal is moving. \\

In a study by Bauman et al. \cite{BAUMAN}, different measurement techniques have been compared while using a treadmill. They quantified kinematic errors due to skin moving artefacts by comparing different techniques on only one run with a high-speed x-ray camera.

Three measuring methods were used to quantify the skin movement error: direct skeletal tracking (bone-derived kinematics), tracking of markers placed on the skin (skin-derived kinematics) and skin-derived kinematics with a calculated estimate of the knee position (triangulated kinematics), these are performed simultaneously. The skin error is provided by comparing the bone-derived kinematics to one of the two skin marker kinematics. For the experiments, they used a treadmill oriented perpendicular to the x-ray video system. The camera records the visible lights converted from the x-rays through an image intensifier. Distortion was controlled by using a calibration frame with lead beads, the calibration was performed before the experiment. It has to be noted that the imqge received is in 2D.

The images of the selected stride are processed, the eventual distortion was corrected and contrast was enhanced. For each analysis, the images were separately corrected. The contrast was even more enhanced for the visualisation of the anatomical landmarks for the bone-derived kinematics and for the visualisation of the markers on the skin. The markers could be automatically tracked with a MATLAB\textsuperscript{\textregistered} software but the anatomical landmarks had to be manually identified frame-by-frame. For the triangulated kinematics, the inputs were the sagittal plane skin-derived positions of the hip and ankle. Then, the position of the knee joint was estimated by drawing a circle with the radius of the femur with its centre on the great trochanter (position of the hip) and another circle with radius of the tibia with its centre on the lateral malleolus (ankle position). The intersection of the two circles was defined as the position of the knee joint. The two skin-derived methods share the same position of the ankle and hip, but do not share the same knee joint position. All of the data was processed with a custom MATLAB\textsuperscript{\textregistered} code.

The results of the experiments were the following. The angles were plotted and we could qualitatively see that the three methods show different results. The differences between the bone-derives and skin-derived kinematics were particularly strong at the location of the hip and knee joint. The greatest error was at the moment the paw had contact with the ground. The skin derived kinematics consistently over-estimated the hip angle values and indicated a reduced range of motion in the hip. For the knee, its angles are also over-estimated but indicate an exaggerated range of motion. The triangulated kinematics show more accurate joint angles than the skin-derived kinematics, even if some error remain.

As a conclusion to this study, the inter-trial variability is ruled out because the kinematics are computed on the same stride. As the differences were significant between bone and skin-derives kinematics, it is safe to say that there is significant skin movement that alters the accurate estimation of the joint angles. 

\textcolor{red}{Here, I have an article about day-to-day reliability of gait characteristics in rats, but it's from 2018 and used markers on skin (big errors as told above) --> use anyway?}

\section{XROMM}